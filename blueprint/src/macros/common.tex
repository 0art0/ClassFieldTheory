% In this file you should put all LaTeX macros and settings to be used both by
% the pdf version and the web version.
% This should be most of your macros.

% The theorem-like environments defined below are those that appear by default
% in the dependency graph. See the README of leanblueprint if you need help to
% customize this.
% The configuration below use the theorem counter for all those environments
% (this is what the [theorem] arguments mean) and never resets it.
% If you want for instance to number them within chapters then you can add
% [chapter] at the end of the next line.
\newtheorem{theorem}{Theorem}
\newtheorem{proposition}[theorem]{Proposition}
\newtheorem{lemma}[theorem]{Lemma}
\newtheorem{corollary}[theorem]{Corollary}
\newtheorem{example}[theorem]{Example}

\theoremstyle{definition}
\newtheorem{definition}[theorem]{Definition}
\newtheorem*{remark}{Remark}

\newcommand{\A}{{\mathbb A}}
\newcommand{\C}{{\mathbb C}}
\newcommand{\F}{{\mathbb F}}
\newcommand{\N}{{\mathbb N}}
\newcommand{\Q}{{\mathbb Q}}
\newcommand{\R}{{\mathbb R}}
\newcommand{\Z}{{\mathbb Z}}

\newcommand{\cO}{{\mathcal O}}

\newcommand{\ab}{\mathrm{ab}}
\newcommand{\aug}{\mathrm{aug}}
\newcommand{\coind}{\mathrm{coind}}
\newcommand{\Cl}{\mathrm{Cl}}
\newcommand{\cor}{\mathrm{cor}}
\newcommand{\density}{\mathrm{density}}
\newcommand{\down}{\mathrm{down}}
\newcommand{\Gal}{\mathrm{Gal}}
\newcommand{\gen}{\mathrm{gen}}
\newcommand{\Hom}{\mathrm{Hom}}
\newcommand{\id}{\mathrm{id}}
\newcommand{\image}{\mathrm{im}}
\newcommand{\ind}{\mathrm{ind}}
\newcommand{\inv}{\mathrm{inv}}
\newcommand{\infl}{\mathrm{infl}}
\newcommand{\leftRegular}{\mathrm{leftRegular}}
\newcommand{\map}{\mathrm{map}}
\newcommand{\reciprocity}{\mathrm{reciprocity}}
\newcommand{\rest}{\mathrm{rest}}
\newcommand{\single}{\mathrm{single}}
\newcommand{\Span}{\mathrm{Span}}
\newcommand{\Split}{\mathrm{split}}
\newcommand{\Tate}{\mathrm{Tate}}
\newcommand{\up}{\mathrm{up}}
\newcommand{\upSes}{\mathrm{upSes}}

% \newcommand{\Inf}{\mathbf{inf}}
\newcommand{\Invar}{\mathbf{invar}}
\newcommand{\Mod}{\mathbf{Mod}}
\newcommand{\Res}{\mathbf{res}}
\newcommand{\Rep}{\mathbf{Rep}}
