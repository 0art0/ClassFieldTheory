% In this file you should put the actual content of the blueprint.
% It will be used both by the web and the print version.
% It should *not* include the \begin{document}
%
% If you want to split the blueprint content into several files then
% the current file can be a simple sequence of \input. Otherwise It
% can start with a \section or \chapter for instance.


\chapter{Group Cohomology}

\section{Group cohomology and homology}

Let $G$ be a group and $R$ a commutative ring.
By a $G$-module, we shall mean an $R$-module $A$ with an action of $G$ by $R$-linear maps.
If $S$ is a subgroup of $G$, then $M \down S$ will mean the same $R$-module $M$, but regarded only as a
representation of $S$.
We shall write $M^S$ for the $R$-submodule of $S$-invariant vectors in $M$.
If the subgroup $S$ is normal then $M^S$ is a representation of the quotient group $G/S$.

Given a represntation $M$ of $G$, there is a cochain comple of $R$-modules
\[
	C^0(G,M) \stackrel{d_0}\to C^1(G,M)  \stackrel{d_1}\to C^2(G,M)  \stackrel{d_2}\to \cdots
\]
where each term $C^n(G,M)$ is the space of functions $G^n \to M$ with some appropriately defined
linear maps $d_i$.
The zeroth module $C^0(G,M)$ should be interpreted as just $M$.
The cochain complex $C^\bullet(G,M)$ is functorial in $M$ and is defined in Mathlib as
\texttt{groupCohomology.cochainsFunctor}.
The cohomology groups of this chain complex are called the cohomology groups of the
$G$-module $M$, and are written $H^n(G,M)$. These are defined in Mathlib as
\texttt{groupCohomology M n}, or \texttt{(groupCohomology.functor n).obj M}.

If $H$ is a subgroup of $G$, then it is common to write $H^n(H,M)$ for the
cohomology groups of $M \down H$.

\begin{definition}
	\label{def:acyclic}
	\lean{Rep.IsAcyclic}
	\leanok
	A representation $M$ of $G$ is called \emph{acyclic} if for every subgroup $S \le G$
	and every $n > 0$, $H^n(S,M) = 0$.
\end{definition}


The following results are in a current PR.

\begin{lemma}
  \label{lem:cochainsFunctor exact}
  The functor taking $M$ to $C^\bullet (G,M)$ is exact.
  I.e. if $0 \to A \to B \to C \to 0$ be a short exact sequence of $G$-modules.
  Then the corresponding sequence of cochain complexes is exact:
  \[
    0 \to C^n(G,A) \to C^n(G,B) \to C^n(G,C) \to 0.
  \]
\end{lemma}

As a consequence of this, we have the following (also a current PR):

\begin{definition}[Long exact sequences]
  \label{def:cohomology Long exact sequence}
  \lean{groupCohomology.δ,
		groupCohomology.longExactSequence₁,
		groupCohomology.longExactSequence₂,
		groupCohomology.longExactSequence₃}
  \uses{lem:cochainsFunctor exact}
  Given a short exact sequence $0 \to A \stackrel{f}\to B \stackrel{g}\to C \to 0$ of $G$-modules,
  the corresponding sequence of cochain complexes is exact:
  $0 \to C^n(G,A) \to C^n(G,B) \to C^n(G,C) \to 0$.
  This implies that there exist "connecting homomorphisms" $\delta : H^n(G,C) \to H^{n+1}(G,A)$,
  such that the following is a long exact sequence:
  \[
    0 \to H^0(G,A) \stackrel{f_*}\to H^0(G,B) \stackrel{g_*}\to H^0(G,C) \stackrel{\delta}\to
    H^1(G,A) \stackrel{f_*}\to H^1(G,B) \stackrel{g_*} \to H^1(G,C) \stackrel{\delta}\to \cdots.
  \]
\end{definition}


\begin{definition}
	\label{def:group homology}
	\lean{groupHomology.inhomogeneousChains,
		groupHomology,
		groupHomology.chainsMap,
		groupHomology.Map}
	There is also a chain complex $R$-modules:
	\[
		\cdots \to C_2(G,M) \to C_1(G,M) \to C_0(G,M)
	\]
	whose $n$-th term is the space of finitely supported functions
	$f : G^n \to_0 M$, with appropriately defined boundary maps.
	In the case $n=0$ this is interpreted as meaning $C_0(G,M) = M$.
	The homology groups of $C_n(G,M)$ are called the \emph{homology groups of $M$}
	and are written $H_n(G,M)$.
	These are not yet in Mathlib but are defined in a current PR.
\end{definition}


\begin{definition}
	\label{def:homology-acyclic}
	\lean{Rep.IsHomologyAcyclic}
	\uses{def:group homology}
	\leanok
	A representation $M$ of $G$ is called \emph{homology-acyclic} if for every subgroup $S \le G$
	and every $n > 0$, $H_n(S,M) = 0$.
\end{definition}



We also note the following, which is a current PR (stated for group homology):

\begin{lemma}
	\label{lem:homology 1 Z}
	\lean{groupHomology.one_trivial_int_iso}
	\leanok
	There is a canonical isomorphism $H_{1}(G,\Z) \cong G^{ab}$.
\end{lemma}


\section{Tate Cohomology}

\begin{definition}
	\label{def:norm}
	\lean{Representation.norm,
		groupCohomology.TateNorm}
	Let $G$ be a finite group and $M$ a representation of $G$ over a commutative ring $R$.
	There is a canonical morphism of representations $N_G : M \to M$ called the \emph{norm},
	defined by
	\[
		N_G(m) = \sum_{x \in G} g \bullet m.
	\]
	We shall also regard the norm as a linear map from $C_0(G,M)$ to $C^0(G,M)$, both of which are
	$M$, and in this context we'll call it the Tate norm.
\end{definition}


\begin{lemma}
	\label{lem:norm comm}
	\lean{Representation.norm_comm,
		Representation.norm_comm'}
	\uses{def:norm}
	\leanok
	For any $g \in G$ and $m \in M$ we have $g \bullet N_G m = N_G m$
	and $N_G (g \bullet m) = N_G m$.
\end{lemma}



\begin{lemma}
	\label{lem:norm comp d}
	\lean{groupCohomology.TateNorm_comp_d}
	\uses{def:norm, lem:norm comm}
	\leanok
	The composition $d^0 \circ N_G$ is zero.
\end{lemma}

\begin{proof}
	The map $d^0 : M \to (G \to M)$ is given by
	$(d^0 m)(g) = m - g\bullet m$.
	Using this formula, we obtain (by \ref{lem:norm comm})
	$d^0 (N_G m) (g) = N_G m - g \bullet N_G m = 0$.
\end{proof}


\begin{lemma}
	\label{lem:d comp norm}
	\lean{groupCohomology.d_comp_TateNorm}
	\uses{def:norm, lem:norm comm, def:group homology}
	The composition $N_G \circ d_0$ is zero.
\end{lemma}

\begin{proof}
	This will follow in a similar way to the previous result
	once group homology is merged into Mathlib.
\end{proof}




\begin{definition}
	\label{def:Tate complex}
	\lean{groupCohomology.TateComplex,
		groupCohomology.TateComplexFunctor}
	\uses{def:group homology,
		lem:norm comp d,
		lem:d comp norm}
	\leanok
	We have a cochain complex $C^n(G,M)$ whose cohomology is $H^n(G,M)$.
	We also have a chain complex $C_n(G,M)$ whose homology is $H_n(G,M)$.
	The terms $C^0(G,M)$ and $C_0(G,M)$ may both be identified with $M$.
	Gluing the chain complex and the cochain complex together with the map $N_G : M \to M$, we obtain
	a cochain complex indexed by $\Z$:
	\[
		\cdots \to C_2(G,M) \to C_1(G,M) \stackrel{d_0}\to C_0(G,M)
		\stackrel{N_G}\to C^0(G,M) \stackrel{d^0}\to C^1(G,M) \to C^2(G,M) \to \cdots
	\]
	We shall write $C^n_{Tate}$ for this cochain, and we normalize the indices so that for
	natural numbers $n$ we have $C^n_{Tate}(G,M) = C^n(G,M)$. This implies
	$C^{-n-1}_{Tate}(G,M) = C_n(G,M)$.
	This construction is functorial in $M$.
\end{definition}

\begin{lemma}
	\label{lem:Tate complex exactness}
  \lean{groupCohomology.TateComplexFunctor_preservesFiniteLimits,
    groupCohomology.TateComplexFunctor_preservesFiniteColimits}
	\uses{def:Tate complex,
		lem:cochainsFunctor exact}
	\leanok
  $C^\bullet_{Tate}$ is an exact functor from $Rep(R,G)$ to the category
  of cochain complexes of $R$-modules.
\end{lemma}

\begin{proof}
	This should follow from the fact that $C^n(G,-)$ and $C_n(G,_)$ are exact functors.
\end{proof}


\begin{definition}
	\label{def:Tate cohomology}
	\lean{groupCohomology.TateCohomology}
	\uses{def:Tate complex}
	For an integer $n$, we shall write $H^n_{Tate}(G,M)$ for the $n$-th cohomology of the complex
	$C^n_{Tate}(G,M)$; this is called the $n$-th Tate cohomology of $M$,
	and is often written $\hat H^n(G,M)$ or just $H^n(G,M)$ in the literature.
	We define this as a functor from the category of representations of $G$ over $R$
	to the category of $R$-modules.
\end{definition}


\begin{definition}
	\label{def:Tate long exact sequence}
	\lean{groupCohomology.TateCohomology.δ}
	\uses{def:Tate cohomology,
		lem:Tate complex exactness}
	\leanok
	Given any short exact sequence of representations
	\[
		0 \to A \to B \to C \to 0,
	\]
	there are connecting homomorphisms $\delta : H^n_{Tate}(G,C) \to H^{n+1}(G,A)$
	such that the following is a long exact sequence:
	\[
		\cdots \to H^n_{Tate}(G,A) \to H^n_{Tate}(G,B) \to H^n_{Tate}(G,C)
		\to H^{n+1}(G,A) \to H^{n+1}_{Tate}(G,B) \to \cdots
	\]
	By the lemma, every short exact sequence of representations gives rise to a long exact sequence
of Tate cohomology groups.
\end{definition}

Our choice of indexing implies the following:

\begin{lemma}
	\label{lem:Tate cohomology is cohomology or homology}
  \lean{groupCohomology.TateCohomology.iso_groupCohomology,
    groupCohomology.TateCohomology.iso_groupHomology}
	\uses{def:Tate cohomology}
	\leanok
	For all natural numbers $n$ we have
	\[
		H^{n+1}_{Tate}(G,M) \cong H^{n+1} (G,M), \qquad
		H^{-n-2}_{Tate}(G,M) \cong H_{n-1} (G,M).
	\]
\end{lemma}

The lemma above does not describe the groups $H^0_{Tate}$ or $H^{-1}_{Tate}$.
They are described by the next two results.

\begin{lemma}
	\label{lem:Tate 0}
	\lean{groupCohomology.TateCohomology_zero_iso,
		groupCohomology.TateCohomology_zero_iso_of_isTrivial}
	\uses{def:Tate cohomology}
	\leanok
	$H^0_{Tate}(G,M) \cong M^G / N_G(M)$.
	In particular if $M$ is a trivial representation of $G$ then
	$H^0_{Tate}(G,M) \cong M / |G|M$.
\end{lemma}

\begin{proof}
	This cannot be done without the definition of group homology.
\end{proof}

\begin{lemma}
	\label{lem:Tate -1}
	\lean{groupCohomology.TateCohomology_neg_one_iso,
		groupCohomology.TateCohomology_neg_one_iso_of_isTrivial}
	\uses{def:Tate cohomology}
	\leanok
	$H^{-1}_{Tate}(G,M) \cong \ker (N_G : M \to M ) /
	\langle {m - g \bullet m : m \in M, g \in G}\rangle$.
	In particular if $M$ is a trivial representation of $G$ then
	$H^{-1}_{Tate}(G,M) \cong M[|G|]$.
\end{lemma}

\begin{proof}
	This cannot be done without the definition of group homology.
\end{proof}



\section{Coinduced and induced representations}

\begin{definition}
	\label{def:coind₁}
	\lean{
		Representation.coind₁,
		Rep.coind₁_obj,
		Rep.coind₁
	}
	\leanok
	Let $G$ be a group and $R$ a commutative ring.
	For an $R$-module $A$, there is an representation of $G$ over $R$
	on the space of all functions $f : G \to A$. The action of an element $g \in G$ on $f$
	is defined by
	\[
		(g \bullet f) (x) = f(xg).
	\]
	This representation is called the coinduced representation and is denoted $\coind_1(G,A)$.
	In fact $\coind_1(G,-)$ is a functor from the category of $R$-modules to the category of
	representations
	of $G$ over $R$.
\end{definition}

\begin{lemma}
	\label{lem:coind₁ acyclic}
	\lean{Rep.coind₁_isAcyclic}
	\uses{def:acyclic,
		def:coind₁}
	\leanok
	The representation $\coind_1(G,A)$ is acyclic. This means that
	for any subgroup $H$ of $G$, the cohomology groups $H^{n+1}(H,\coind_1(G,A))$ are
	all zero.
\end{lemma}

\begin{proof}
	There is an elementary proof outlined in lean file.
\end{proof}

\begin{lemma}
	\label{lem:coind₁ invariants}
	\lean{Rep.coind₁_quotientToInvariants_iso}
	\uses{def:coind₁}
	\leanok
	Let $H$ be a normal subgroup of $G$. Then $\coind_1(G,A)^H \cong \coind_1(G/H,A)$.
\end{lemma}

\begin{corollary}
	\label{cor:coind₁ invariants acyclic}
	\lean{Rep.coind₁_quotientToInvariants_isAcyclic}
	\uses{lem:coind₁ invariants,
		lem:coind₁ acyclic}
	\leanok
	Let $H$ be a normal subgroup of $G$ and let $M^H$ be the representation of $G/H$ on the
	$H$-invariant vectors in $M$. Then $\coind_1(G,M)^H$ is acyclic as a representation of $G/H$.
\end{corollary}

\begin{proof}
	\leanok
	This follows directly from Lemmas \ref{lem:coind₁ invariants} and \ref{lem:coind₁ acyclic}.
\end{proof}

\begin{definition}
	\label{def:ind₁}
	\lean{Rep.ind₁}
	\leanok
	Let $G$ be a group and $R$ a commutative ring.
	For an $R$-module $A$, there is an representation of $G$ over $R$
	on the space of all finitely supported functions $f : G \to A$.
	The action of an element $g \in G$ on $f$ is defined by
	\[
		(g \bullet f) (x) = f(g^{-1}x).
	\]
	This representation is called the induced representation and is denoted $\ind_1(G,A)$.
	In fact $\ind_1(G,-)$ is a functor from the category of $R$-modules to the category of
	representations	of $G$ over $R$.
\end{definition}

\begin{lemma}
	\label{lem:ind₁ homology-acyclic}
	\lean{Rep.ind₁_isHomologyAcyclic}
	\uses{def:ind₁,
		def:homology-acyclic}
	\leanok
	The representation $\ind_1(G,A)$ is homology-acyclic. This means that
	for any subgroup $H$ of $G$, the homology groups $H_{n+1}(H,\ind_1(G,A))$ are
	all zero.
\end{lemma}

\begin{proof}
	There is an elementary proof, but this cannot be inserted until
	the PR on group homology is merged.
\end{proof}

\begin{definition}
	\label{def:ind₁ to coind₁}
	\lean{
		Representation.ind₁_toCoind₁,
		Rep.ind₁_toCoind₁}
	\uses{def:coind₁, def:ind₁}
	\leanok
	There is a morphism of representations $\ind_1(G,A) \to \coind_1(G,A)$,
	which takes a finitely supported function $f : G \to_0 A$ to the function
	\[
		x \mapsto f(x^{-1}).
	\]
	This map is an isomorphism of representations if the group $G$ is finite.
\end{definition}

\begin{lemma}
	\label{lem:ind₁ to coind₁}
	\lean{
		Representation.ind₁_toCoind₁,
		Representation.ind₁_toCoind₁_comm,
		Representation.ind₁_equiv_coind₁,
		Rep.ind₁_toCoind₁,
		Rep.instIsIsoAppModuleCatInd₁_toCoind₁OfFinite,
		Rep.ind₁_iso_coind₁}
	\uses{def:ind₁ to coind₁}
	\leanok
	The map $\ind_1(G,A) \to \coind_1(G,A)$ is a morphism of representations,
	and is an isomorphism if $G$ is a finite group.
\end{lemma}


\begin{corollary}
	\label{cor:ind₁ is acyclic}
	\lean{Rep.ind₁_isAcyclic,
		Rep.coind₁_isHomologyAcyclic}
	\uses{lem:ind₁ to coind₁,
		lem:coind₁ acyclic,
		lem:ind₁ is homology-acyclic}
	If the group $G$ is finite then $\coind_1(G,A)$ is homology-acyclic
	and $\ind_1(G,A)$ is acyclic.
\end{corollary}

\begin{proof}
	This follows directly from lemmas \ref{lem:ind₁ to coind₁},	\ref{lem:coind₁ acyclic}
		and \ref{lem:ind₁ is homology-acyclic}.
	\leanok
\end{proof}


\section{Dimension-shifting}

\subsection{Shifting $\up$}

\begin{definition}
	\label{def:coind₁'}
	\lean{Representation.coind₁',
		Rep.coind₁'}
	Let $G$ be a group and $M$ a representation of $G$ over a commutative ring $R$.
	There is a representation $\coind_1' (M)$ on the $R$-module of
	functions $G \to M$.
	The action of an element $g \in G$ on a function $f : G \to M$ is given by
	\[
		(g \bullet f)(x) = g \bullet (f (xg)).
	\]
\end{definition}

\begin{lemma}
	\label{lem:coind₁' iso coind₁}
	\lean{Rep.coind₁'_obj_iso_coind₁}
	\uses{def:coind₁, def:coind₁'}
	\leanok
	The representations $\coind_1'(M)$ and $\coind_1(G,M)$ of $G$ are isomorphic.
\end{lemma}

\begin{corollary}
	\label{cor:coind₁' acyclic}
	\lean{Rep.coind₁'_isAcyclic}
	\uses{lem:coind₁' iso coind₁,
		lem:coind₁ acyclic}
	\leanok
	The representation $\coind_1'(M)$ is acyclic.
\end{corollary}

\begin{proof}
	This follows directly from Lemmas \ref{lem:coind₁' iso coind₁} and \ref{lem:coind₁ acyclic}.
	\leanok
\end{proof}

\begin{corollary}
	\label{cor:coind₁' invariants acyclic}
	\lean{Rep.coind₁'_quotientToInvariants_isAcyclic}
	\uses{lem:coind₁ invariants acyclic,
		lem:coind₁' iso coind₁}
	Let $H$ be a normal subgroup of $G$. Then $\coind_1'(M)^H$ is an acyclic representation of $G/H$.
\end{corollary}

\begin{proof}
	We've seen in Lemma \ref{lem:coind₁' iso coind₁} that $\coind_1'(M)$
	is isomorphic to $\coind_1(M)$.
	We will need to define the $H$-invariants (Rep.quotientToInvariants)
	as a functor from $\Rep(R,G)$ to $\Rep(R,G/H)$.
	Then by functoriality, $\coind_1'(M)^H$ is isomorphic to $\coind_1(M)^H$,
	which is already known to be acyclic (Lemma \ref{lem:coind₁ acyclic}).
\end{proof}


\begin{definition}
	\label{def:coind₁'_ι}
	\lean{Representation.coind₁'_ι}
	\uses{def:coind₁'}
	\leanok
	There is a morphism $M \to \coind_1'(M)$ which takes a vector $m \in M$
	to the constant function on $G$ with value $m$.
\end{definition}

\begin{lemma}
	\label{lem:coind₁'_ι natural}
	\lean{Rep.coind₁'_ι}
	\uses{def:coind₁'_ι}
	\leanok
	The map $M \to \coind_1'(M)$ is a natural transformation from the identity
	functor to $\coind₁'$.
\end{lemma}

\begin{lemma}
	\label{lem:coind₁'_ι mono}
	\lean{Rep.dimensionShift.instMonoAppCoind₁'_ι}
	\uses{def:coind₁'_ι}
	\leanok
	The map $M \to coind₁'(M)$ is a monomorphism, i.e. injective.
\end{lemma}

\begin{proof}
  The function which takes an element $m \in M$ to the constant
  function on $G$ with value $m$ is clearly injective.
\end{proof}

\begin{definition}
	\label{def:up}
	\lean{Rep.dimensionShift.up}
	\uses{def:coind₁'_ι}
	We define $\up(M)$ to be the cokernel of the map $M \to \coind_1'(M)$,
	and show that $\up$ is a functor from $\Rep(R,G)$ to $\Rep(R,G)$.
\end{definition}

\begin{definition}
	\label{def:up ses}
	\lean{Rep.dimensionShift.upSes,
		Rep.dimensionShift.up_shortExact,
		Rep.dimensionShift.up_shortExact_res}
	\uses{def:up}
	\leanok
	We define a functor $\upSes$ which takes a representation $M$ to the complex
	\[
		0 \to M \to \coind_1'(M) \to \up(M) \to 0.
	\]
	This is a short exact sequence in $\Rep(R,S)$ for any subgroup $S \le G$.
\end{definition}

\begin{corollary}
	\label{cor:up iso}
	\lean{Rep.dimensionShift.up_δ_zero_epi,
		Rep.dimensionShift.up_δ_isIso,
		Rep.dimensionShift.up_δis0,
		Rep.dimensionShift.up_δ_zero_epi_res,
		Rep.dimensionShift.up_δ_isIso_res,
		Rep.dimensionShift.up_δiso_res}
	\uses{def:up ses,
		cor:coind₁' acyclic}
	\leanok
	Let $S$ be a subgroup of $G$ and let $n \in \N$.
	Then the connecting map from the long exact sequence $H^{n+1}(S,\up(M)) \to H^{n+2}(S,M)$ is an
	isomorphism.
	The corresponding map $H^{0}(S,\up(M)) \to H^{1}(S,M)$ is	surjective.
\end{corollary}

\begin{proof}
	We have already shown in Corollary \ref{cor:coind₁' acyclic}
	that $\coind_1'(M)$ is acyclic, so $H^{r}(S,\coind_1'(M))=0$
	for all $r>0$.
\end{proof}

\begin{lemma}
	\label{lem:up iso natural}
	\lean{Rep.dimensionShift.up_δiso_natTrans}
	\uses{cor:up iso}
	\leanok
	The isomorphism $H^{n+1}(G,\up(-)) \cong H^{n+2}(G,-)$ is an isomorphism of functors.
	This means that for every morphism $f : M \to N$ of representations, the following square commutes:
	\[
		\begin{matrix}
			H^{n+1}(G,\up(M)) & \cong & H^{n+2}(G,M) \\
			\down && \down \\
			H^{n+1}(G,\up(N)) & \cong & H^{n+2}(G,N)
		\end{matrix}.
	\]
\end{lemma}

\begin{proof}
	This follows from \texttt{HomologicalComplex.HomologySequence.δ_naturality}
	because the short exact sequence $0 \to M \to \coind_1'(M) \to \up(M) \to 0$
	is functorial in $M$.
\end{proof}

\subsection{Shifting $\down$}
Let $G$ be a group and $M$ a representation of $G$ over a commutative ring $R$.

\begin{definition}
	\label{def:ind₁'}
	\lean{Representation.ind₁',
		Rep.ind₁'}
	There is a representation $\ind_1' (M)$ on the $R$-module of finitely supported
	functions $G \to_0 M$.
	The action of an element $g \in G$ on a function $f : G \to_0 M$ is given by
	\[
		(g \bullet f)(x) = g \bullet (f (g^{-1}x)).
	\]
	The map $\ind_1'$ is functorial in $M$.
\end{definition}


\begin{lemma}
	\label{lem:ind₁' iso ind₁}
	\lean{
		Representation.ind₁'_lequiv,
		Representation.ind₁'_lequiv_comm,
		Rep.ind₁'_obj_iso,
		Rep.ind₁'_iso_forget₂_ggg_ind₁,
	}
	\uses{def:ind₁',
		def:ind₁}
	\leanok
	The representations $\ind_1'(M)$ and $\ind_1(G,M)$ are isomorphic; more precisely the
	functors $\ind_1'$ and $\ind_1(G,-)$ are isomorphic.
\end{lemma}

\begin{proof}
	The data of the isomorphism is contained in the lean file; the isomorphism
	takes $f : G \to_0 M$ to the finitely supported function
	\[
		x \mapsto x^{-1} \bullet f(x).
	\]
	It remains to check linearity and naturality.
\end{proof}

\begin{corollary}
	\label{cor:ind₁' homology-acyclic}
	\lean{Rep.ind₁'_isHomologyAcyclic}
	\uses{lem:ind₁' iso ind₁,
		lem:ind₁ homology-acyclic}
	\leanok
	The representation $\ind_1'(M)$ is homology-acyclic.
\end{corollary}

\begin{proof}
	\leanok
	We've shown that $\ind_1'(M)$ is isomorphic to $\ind_1(M)$, which is already known to
	be homology-acyclic.
\end{proof}


\begin{corollary}
	\label{cor:ind₁' iso coind₁'}
	\lean{Rep.ind₁'_iso_coind₁',
		Rep.ind₁'_isAcyclic,
		Rep.coind₁'_isHomologyAcyclic}
	\uses{lem:ind₁' iso ind₁,
		lem:ind₁ to coind₁,
		lem:coind₁' iso coind₁}
	\leanok
	If the group $G$ is finite then $\ind_1'(M) \cong \coind_1'(M)$.
	In particular $\coind_1'(M)$ is homology-acyclic and $\ind_1'(M)$ is acyclic.
\end{corollary}

\begin{proof}
	Since $G$ is assumed to be finite we have an isomorphism $\ind_1 \cong \coind_1$.
	\leanok
\end{proof}

\begin{definition}
	\label{def:ind₁'_π}
	\lean{Representation.ind₁'_π,
		Representation.ind₁'_π_comm}
	\uses{def:ind₁'}
	For any representation $M$, there is a morphism $\ind_1'(M) \to M$,
	which takes a finitely supported function $f : G \to_0 M$ to the sum $\sum_{x \in G} f (x)$.
\end{definition}

\begin{lemma}
	\label{lem:ind₁'_π}
	\lean{Rep.ind₁'_π}
	\uses{def:ind₁'_π}
	\leanok
	More precisely the map $\ind_1'(M) \to M$ is a morphism of functors from $\ind_1'$
	to the identity functor on $\Rep(R,G)$.
	This means that for every morphism $f : M \to N$ of representations, the following square
	commutes
	\[
		\begin{matrix}
			\ind_1'(M) & \to & \ind_1'(N) \\
			\down && \down \\
			M & \to &N
		\end{matrix}.
	\]
\end{lemma}

\begin{definition}
	\lean{
		Rep.dimensionShift.down,
		Rep.dimensionShift.down_ses,
		Rep.dimensionShift.down_shortExact,
		Rep.dimensionShift.down_shortExact_res,
		Rep.dimensionShift.down_δ_zero_epi,
		Rep.dimensionShift.down_δ_zero_res_epi,
		Rep.dimensionShift.down_δ_isIso,
		Rep.dimensionShift.down_δiso,
		Rep.dimensionShift.down_δiso_natTrans,
		Rep.dimensionShift.down_δ_res_isIso,
		Rep.dimensionShift.down_δiso_res
	}
	There is an surjection $\ind_1'(M) \to M$ which takes a finitely supported
	function $f : G \to_0 M$ to the sum $\sum_{x \in G} f (x)$.
	We define $\down(M)$ to be the kernel of this map.
	We therefore have a short exact sequence of representations.
	\[
		0 \to \down(M) \to \ind_1'(M) \to M \to 0.
	\]
	By the lemma, we have for any subgroup $H$ of $G$ and any $n \in \N$ an isomorphism:
	\[
		H_{n+1}(H,\down(M)) \cong H_{n+2}(H,M).
	\]
	It is convenient to define $\ind_1'$ and $\down$ as functors, and the isomorphism
	above as an isomorphism of functors.
\end{definition}


\begin{lemma}
	\lean{groupCohomology.TateCohomology_coind₁,
		groupCohomology.TateCohomology_ind₁'}
	If $M$ is a representation of a finite group $G$ then the representations
	$\ind_1'(M)$ and $\coind_1'(M)$ are ``Tate-acyclic''. This means that for all
	subgroups $H$ of $G$ and all integers $n$, we have
	$H^n_{Tate}(H,\coind_1'(M)) = 0$ and $H^n_{Tate}(H,\ind_1'(M)) = 0$.
\end{lemma}

\begin{proof}
	By what we've already proved, it's sufficient to prove this in the cases $n = 0$ and $n = -1$.
	These cases can be done by hand.
\end{proof}


% \subsection{Finite groups}

% Suppose now that the group $G$ is finite.
% In this case, there is an isomorphism or representations $\ind_1'(M) \cong \coind_1'(M)$ which takes $f : G \to_0 M$ to the function
% \[
% 	x \mapsto f(x^{-1}).
% \]
% As a consequence, the representations $\coind_1'(M)$ and $\ind_1'(M)$ are both
% acyclic and homology-acyclic.

% \begin{lemma}
% 	If $M$ is a representation of a finite group $G$ then the representations
% 	$\ind_1'(M)$ and $\coind_1'(M)$ are ``Tate-acyclic''. This means that for all
% 	subgroups $H$ of $G$ and all integers $n$, we have
% 	$H^n_{Tate}(H,\coind_1'(M)) = 0$ and $H^n_{Tate}(H,\ind_1'(M)) = 0$.
% \end{lemma}

% \begin{proof}
% 	By the remark above, it's sufficient to prove this in the cases $n = 0$ and $n = -1$.
% \end{proof}

\begin{corollary}
	\lean{groupCohomology.instIsIsoModuleCatδ,
		groupCohomology.instIsIsoModuleCatδ_1,
		groupCohomology.upδiso_Tate,
		groupCohomology.downδiso_Tate
	}
	If the group $G$ is finite then for every subgroup $H$ of $G$
	and every $n \in \Z$ we have isomorphisms
	\[
		H^n_{Tate}(H, \up(M)) \cong H^{n+1}_{Tate}(H,M),
		\qquad
		H^{n+1}_{Tate}(H, \down(M)) \cong H^{n}_{Tate}(H,M).
	\]
\end{corollary}





\section{The inflation-restriction sequence}


\begin{definition}
	\lean{groupCohomology.infl}
	Let $H$ be a normal subgroup of a group $G$ and $M$ is
	a representation of $G$ (over some commutative ring $R$).
	The subspace $M^H = \{m \in M | \forall h \in H, h \bullet m = 0\}$ is
	a representation of the quotient group $G / H$.
	Any n-chain $\sigma \in C^n(G/H,M^H)$ may be pulled back to an
	$n$-chain on $G$ with values in $n$. This defines a map of complexes
	\[
		C^\bullet(G/H, M^H) \to C^\bullet(G,H).
	\]
	The corresponding map $\infl : H^\bullet(G/H, M^H) \to H^\bullet(G,M)$
	is called the ``inflation map''.
\end{definition}


\begin{definition}
	\lean{groupCohomology.rest}
	The restriction of an $n$-cochain $\sigma\in C^n(G,M)$ to a subgroup $H$
	is an $n$-cochain on $H$. Restriction of cochains gives a map of complexes
	\[
		C^\bullet(G,M) \to C^\bullet(H,M).
	\]
	The corresponding map $\rest : H^\bullet(G,M) \to H^\bullet(H,M)$ is called the
	``restriction map''.
\end{definition}

\begin{theorem}
	\lean{groupCohomology.inflationRestriction,
		groupCohomology.inflation_restriction_mono,
		groupCohomology.inflation_restriction_exact}
	Assume that for all natural numbers $i < n$ we have $H^{i+1}(H,M)=0$.
	Then the following sequence is exact:
	\[
		0 \to H^{n+1}(G/H, M^H) \to H^{n+1}(G,M) \to H^{n+1}(H,M),
	\]
	where the first map is inflation and the second is restriction.
\end{theorem}

\begin{proof}
	This is already in Mathlib for $n=0$, i.e. for cohomology in
	dimension $1$.
	We prove it in general by induction on $n$ using the dimension-shifting function $\up$.
	Recall that we have a short exact sequence of representations of $G$:
	\[
		0 \to M \to \coind_1'(M) \to \up(M) \to 0.
	\]
	By assumption $H^1(H,M)=0$, so by taking $H$-invariants we obtain a short exact sequence of $G/H$-modules:
	\[
		0 \to M^H \to \coind_1'(M)^H \to \up(M)^H \to 0.
	\]
	Corresponding to this short exact sequence we have a long exact sequence in cohomology
	containing the following section
	\[
		 0 \to H^n(G/H, \up(M)^H) \to  H^{n+1}(G/H, M^H) \to 0
	\]
	We have a diagram where the horizontal maps are inflation and restriction maps and the vertical
	maps are isomorphisms.
	\[
		\begin{matrix}
			0 &\to& H^{n}(G/H, \up(M)^H) &\to &H^{n}(G,\up(M)) & \to & H^n(H,\up(M)) \\
			  &   &   \downarrow              &    &  \downarrow         &     &  \downarrow  \\
			0 &\to& H^{n+1}(G/H, M^H) &\to &H^{n+1}(G,M) & \to & H^{n+1}(H,M).
		\end{matrix}
	\]
	The top row is exact by the inductive hypothesis.
	We must show that the second row is exact; this amounts to showing that the diagram commutes.
	This follows from the next two lemmas.
\end{proof}

\begin{lemma}
	\lean{groupCohomology.rest_δ_naturality}
	Suppose we have a a short exact sequence of representations of $G$
	\[
			0 \to  A \to  B \to  C  \to  0.
	\]
	For every subgroup $H$ of $G$ the following square commutes:
	\[
		\begin{matrix}
			H^{n}(G,C) & \to & H^{n+1}(G,A)\\
			\downarrow & & \downarrow \\
			H^n(H,C) & \to &H^{n+1}(H,A)
		\end{matrix}.
	\]
	The vertical maps are restriction maps and the horizontal maps are the connecting homomorphisms
	from the long exact sequence.
\end{lemma}

\begin{lemma}
	\lean{groupCohomology.infl_δ_naturality}
	Let $H$ be a normal subgroup of $G$, and suppose we have a short exact
	sequence of $G$-modules:
	\[
			0 \to  A \to  B \to  C  \to  0.
	\]
	Assume also that the following is a short exact sequence of $G/H$-modules:
	\[
			0 \to  A^H \to  B^H \to  C^H  \to  0.
	\]
	Then for all $n \in \N$ the following square commutes
	\[
		\begin{matrix}
			H^{n}(G/H,C^H) & \to & H^{n+1}(G/H,A^H)\\
			\downarrow & & \downarrow \\
			H^n(G,C) & \to & H^{n+1}(G,A)
		\end{matrix}.
	\]
	The vertical maps are inflation and the horizontal maps are the connecting homomorphisms
	from the long exact sequences.
\end{lemma}


\section{Periodicity for finite cyclic groups}

Suppose now that $G$ is a finite cyclic group of order $n$.
We shall write $g$ for a fixed generator of $G$.
Let $M$ be a representation of $G$.


\begin{definition}
	\lean{Representation.map₁}
	There is a map $\map_1 : \coind_1'(M) \to \coind_1'(M)$ which takes $f : G \to M$ to the function
	\[
		x \mapsto f(x) - f(g^{-1}x).
	\]
\end{definition}


\begin{lemma}
	\lean{Representation.map₁_ker}
	The kernel of $\map_1$ consists of the constant functions $G \to M$, i.e. the image of the
	map $M \to \coind_1'(M)$.
\end{lemma}

\begin{corollary}
	The image of $\map_1$ is isomorphic to $\up(M)$.
\end{corollary}

Recall that since $G$ is finite, the representations $\coind_1'(M)$ and $\ind_1'(M)$
are isomorphic, and we define $\map_2$ to be the corresponding map $\ind_1'(M) \to \ind_1'(M)$.
This is given by
\[
	\map_2(f) (x) = f(x) - f(xg^{-1}).
\]

\begin{lemma}
	\lean{Representation.map₂_range}
	The image of $\map_2 : \ind_1'(M) \to \ind_1'(M)$ is precisely the set of functions $G \to M$
	which sum to zero. This is the kernel of the map $\ind_1'(M) \to M$.
\end{lemma}




The two lemmas show that we have a diagram with exact rows and vertical isomorphisms:
\[
	\begin{matrix}
		0 \to M \to &\coind_1'(M)& \stackrel{\map_1}\to& \coind_1'(M)\\
		& || &&|| \\
		& \ind_1'(M)& \stackrel{\map_2}\to& \ind_1'(M) &\to M \to 0
	\end{matrix},
\]


\begin{corollary}
	\lean{Rep.up_iso_down}
	In particular we have an isomorphism:
	\[
		\up(M) \cong \down(M).
	\]
	(It is more convenient in the long run to express this as an isomorphism of functors
	$\up \cong \down$.)
\end{corollary}

\begin{corollary}
	\lean{Rep.periodicCohomology}
	For all $n \in \N$, $H^{n+1}(G,M) \cong H^{n+3}(G,M)$.
\end{corollary}







\section{The Acyclic Criterion}

Recall that a representation $M$ of a group $G$ is \emph{acyclic} if for all subgroups $H$ of $G$
and all $n \in \N$, the cohomology groups $H^{n+1}(H,M)$ are zero.

\begin{theorem}
	\lean{groupCohomology.Acyclic_of_even_of_odd_of_solvable}
	Let $M$ be a representation of a finite solvable group $G$.
	Suppose we have positive natural numbers $e$ and $o$ with $e$ even and $o$ odd, such that for all subgroups $H$ of $G$ we have
	\[
		H^e(H,M) = 0, \qquad H^o(H,M) = 0.
	\]
	Then $M$ is acyclic.
\end{theorem}

\begin{proof}
	We must prove that $H^{n+1}(H,M) = 0$ for all $H$ and all $n$.
	We'll prove this by induction on $H$. The result is true for the trivial subgroup of $G$.
	Assume that the result is true for $H$, and assume that $H' / H$ is cyclic.
	The inductive hypothesis implies that (for al $n$) the inflation restriction sequence is exact:
	\[
		0 \to H^{n + 1} (H'/H, M^H) \to H^{n+1}(H' , M) \to H^{n+1}(H,M)= 0.
	\]
	We therefore have isomorphisms $H^{n + 1} (H'/H, M^H) \cong H^{n+1}(H' , M)$.
	In particular we have $H^{e} (H'/H, M^H) = 0$ and $H^{o} (H'/H, M^H) = 0$.
	Using periodicity of the cohomology of a cyclic group, we have $H^{n+1}(H'/H,M^H)=0$ for all $n$.
\end{proof}


\begin{theorem}
	\lean{groupCohomology.Acyclic_of_even_of_odd}
	Let $M$ be a representation of a finite group $G$ (no longer assumed to be solvable).
	Suppose we have positive natural numbers $e$ and $o$ with $e$ even and $o$ odd, such that for all subgroups $H$ of $G$ we have
	\[
		H^e(H,M) =0, \qquad H^o(H,M) = 0.
	\]
	Then $M$ is acyclic.
\end{theorem}

\begin{proof}
	(Requires definition of corestriction, which is not yet in Mathlib).
	Let $H$ be a subgroup of $G$.
	Fix a prime number $p$ and let $S_p$ be the Sylow $p$-subgroup of $H$.
	Consider the composition of the restriction and corestriction maps
	\[
		H^{n+1}(H,M) \to H^{n+1}(S_p,M) \to H^{n+1}(H,M).
	\]
	This composition is multiplication by the index $[H:S_p]$,
	and is therefore injective on the $p$-torsion
	in $H^{n+1}(H,M)$.
	However, the previous result implies (since $S_p$ is solvable) that $H^{n+1}(S_p,M)=0$.
	Therefore $H^{n+1}(G,M)$ has zero $p$-torsion.
	The same composition with $S_p$ replaced by the trivial subgroup implies that every element
	of $H^{n+1}(H,M)$ is killed by $|H|$.
	Therefore $H^{n+1}(H,M) = 0$.
\end{proof}


\begin{corollary}
	\lean{Rep.dimensionShift.up_isAcyclic,
		Rep.dimensionShift.down_isAcyclic}
	If $M$ is an acyclic representation of a finite group $G$
	then $\up(M)$ and $\down(M)$ are acyclic.
\end{corollary}

\begin{proof}
	This follows from the previous result, together with the dimension shifting isomorphisms.
\end{proof}

\begin{theorem}
	\lean{groupCohomology.TateCohomology_of_isAcyclic,
		Rep.isHomologyAcyclic_of_isAcyclic}
	Let $G$ be an acyclic representation of a finite group $G$.
	Then for all integers $n$ and all subgroups $H$ of $G$ we have $H^n_{Tate}(H,M)=0$.
	In particular $M$ is homology-acyclic.
\end{theorem}

\begin{proof}
	Fix an integer $n$ and choose a natural number $m$ such that $m + n > 0$.
	By the previous lemma, $\down^m(M)$ is acyclic.
	Therefore
	\[
		H^n_{Tate}(H,M) \cong H^{n+m}(H, \down^m(M)) \cong 0.
	\]
\end{proof}



\section{The augmentation module $\aug(R,G)$}

Let $G$ be a group and $R$ a commutative ring. We shall also write $R$ for the trivial
representation of $G$ on $R$.
The left regular representation of $G$ is the representation $\ind_1'(R)$, whose
vectors consist of finitely supported functions $f : G \to_0 R$.
There is a surjective morphism $\leftRegular(R,G) \to R$ which takes $f$ to $\sum_{x \in G} f(x)$.
The augmentation module $\aug(R,G)$ is defined to be the kernel of this map.
We therefore have a short exact sequence
\[
	0 \to \aug(R,M) \to \leftRegular(R,G) \to R \to 0.
\]
If we assume that $G$ is finite, then we have shown above that $\leftRegular(R,G)$ is acyclic.
In particular we have isomorphisms for all $n \in \Z$ and all subgroups $H$ of $G$:
\[
	H^n_{Tate}(H,R) \cong H^{n+1}_{Tate}(H,\aug(R,G)).
\]


\begin{lemma}
	Let $G$ be a finite group.
	Then for every subgroup $H$ of $G$ we have an isomorphism $H^1(H,\aug(R,G)) = R / |H| R$.
\end{lemma}

\begin{proof}
	Since $R$ is a trivial $H$-module, we have $H^0_{Tate}(H,R) \cong R / |H| R$.
\end{proof}


\begin{lemma}
	Let $G$ be a finite group and assume that $R$ has no additive torsion.
	Then for all subgroups $H$ of $G$ we have $H^2(H,\aug(R,G)) = 0$.
\end{lemma}

\begin{proof}
	Since $R$ has no additive torsion we have
	$H^1(G,R) \cong \Hom(G,R) = 0$.
\end{proof}


\section{The splitting module}

In this section $G$ is a finite group; $M$ is a representation of $G$ over a commutative ring $R$
 and $\sigma \in H^2(G,M)$ is a cohomology class satisfying the following conditions for all
 subgroups $H$ of $G$:
\begin{itemize}
	\item
	$H^1(H,M)=0$.
	\item
	$H^2(H,M)$ isomorphic to $R / |H|R$, and is generated by the restriction $\sigma | H$.
\end{itemize}
We shall also assume that $R$ has no additive torsion. This implies
\[
	H^2(G,\aug(R,G)) \cong H^1(G,R) \cong \Hom(G,R) = 0.
\]


The two conditions together mean that $H^2(H,M)$ is isomorphic to $R/|H|R$.
We shall write $\sigma'$ for an inhomogeneous 2-cocycle representing $\sigma$.

\begin{definition}
	The splitting module of $\sigma'$ is the $R$-module $M \times \aug(R,G)$,
	with the action of an element $g \in G$ given by
	\[
		g \bullet (m,f)
		= \left(g \bullet m + \sum_{x \in G} f(x) \sigma'(g,x) , g \bullet f\right).
	\]
	(Although we don't need this fact right now, it's worth knowing that up to isomorphism, the splitting module depends only on the cohomology class $\sigma$. For this reason, we'll write
	$\Split(\sigma)$ for this representation).
\end{definition}



There is evidently a short exact sequence of representations of $G$.
\[
	0 \to M \to \Split(\sigma) \to \aug(R,G) \to 0.
\]

\begin{lemma}
	The image of $\sigma$ in $H^2(G,\Split(\sigma))$ is zero.
\end{lemma}

\begin{proof}

\end{proof}

\begin{theorem}
	Assume $R$ and $\sigma$ satisfy the conditions above.
	Then	$\Split(\sigma)$ is acyclic.
\end{theorem}

\begin{proof}
	By the acyclicity criterion, it's enough to prove for every subgroup $H$ of $g$ that
	$H^1(H,\Split(\sigma))=0$ and $H^2(H,\Split(\sigma))=0$.
	We have a long exact sequence with the following terms:
	\[
		0 \to H^1(H,\Split(\sigma)) \to H^1(H,\aug(R,G)) \to H^2(H,M) \to H^2(H,\Split(\sigma))
		\to 0.
	\]
	Since $\sigma|H$ generates $H^2(H,M)$, the previous lemma implies that the last map is zero,
	and in particular $H^2(H,\Split(\sigma))=0$ (this uses the fact that restriction is a natural
	transformation).
	The $R$-modules $H^1(H,\aug(R,G))$ and $H^2(H,M)$ are both isomorphic to $R / |H|R$,
	and the map from one to the other is surjective.
	Since every surjective $R$-module endomorphism of $R /|H|R$ is injective, the map
	from $H^1(H,\aug(R,G))$ to $H^2(H,M)$ is injective.
	Therefore $H^1(H,\Split(\sigma))=0$.
\end{proof}



The theorem implies that we have isomorphisms for all $n\in \Z$ (which depend of $\sigma$):
\[
	H^{n}_{Tate}(G,R) \cong H^{n+1}_{Tate}(G,\aug(R,G)) \cong H^{n+2}_{Tate}(G,M).
\]
In particular in the case $n = -2$, $R = \Z$ we have the (inverse of the) reciprocity isomorphism
\[
	G^{ab} \cong H^{-2}(G,\Z) \cong M^G / N_G M.
\]
