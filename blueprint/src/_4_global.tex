\chapter{Global Class Field Theory}

In this chapter we let $l/k$ be a finite Galois extension of
algebraic number fields. We shall consider the idele class group $\Cl_l = \A_l^\times / l^\times$
as a module for the Galois group $\Gal(l/k)$ and we shall describe the construction of
fundamental classes in $H^2(l/k, \Cl_l)$.
These classes give rise to a reciprocity isomorphism
\[
	\Gal(l/k)^{\ab} \cong \Cl_k / N(\Cl_l).
\]
We would therefore like to prove for all intermediate fields $l / m / k$:
\begin{itemize}
	\item
	$H^1(l/m, \Cl_l) = 0$,
	\item
	$H^2(l/m, \Cl_l)$ is cyclic of order $[l:m]$.
\end{itemize}

We note the following consequence of Hilbert's theorem 90:

\begin{lemma} \label{lem:idele class invariants}
	\uses{thm:hilbert 90}
	Let $l/k$ be a finite Galois extension of number fields (or even global fields).
	The map $\Cl_k \to \Cl_l ^ {\Gal(l/k)}$ is an isomorphism.
\end{lemma}

\begin{proof}
	We have a short exact sequence
	\[
		0 \to l^\times \to \A_l^\times \to \Cl_l \to 0.
	\]
	Taking $\Gal(l/k)$-invariants gives a long exact sequence beginning with
	\[
		0 \to k^\times \to \A_k^\times \to \Cl_l^{\Gal(l/k)} \to H^1(l/k, l^\times).
	\]
	The last term is zero by \ref{thm:hilbert 90}. This implies the result.
\end{proof}

By the lemma, we may regard $H^0_{\Tate}(l/k,\Cl_l)$ as the quotient
$\Cl_k / N_{l/k} \Cl_l$.
Furthmore, the inflation map for a tower of Galois extensions $l/m/k$ takes the form
\[
	H^\bullet(m/k, \Cl_m) \to H^\bullet(l/k, \Cl_l).
\]







\section{Choice of $S$}

Let $S$ be a finite set of places of $k$, containing all of the infinite primes and all
of the primes which ramify in $l$.
We shall use the notation
\[
	\A_{l,S} = \prod_{v \in S} \prod_{w | v} l_w \times \prod_{v \notin S} \prod_{w | v} \cO_v.
\]
We also write $\cO_{l,S}$ for the ring $S$-integers in $l$:
\[
	\cO_{l,S} = \{x \in l, \forall v \not\in S, \forall w | v, |x|_v \le 1\} = l \cap \A_{l,S}.
\]
We regard $\cO_{l,S}$ as a subring of $\A_{l,S}$ and
$\cO_{l,S}^\times$ as a subgroup of $\A_{l,S}^\times$.

The quotient $\A_l^\times / \A_{l,S}^\times$ is naturally isomorphic to the
group of non-zero fractional ideals of $\cO_{l,S}$.
By adding more primes to $S$ if necessary, we may assume that $\cO_{k,S}$ and $\cO_{l,S}$ are both
principal ideal domains (this can be achieved by adding to $S$ primes $P$ whose ideal classes
generate the class groups of $k$ and $l$).
With such a choice of $S$ we have:
\[
	\A_{l}^\times = \A_{l,S}^\times l^\times,
	\qquad
	\A_k^\times = \A_{k,S}^\times k^\times,
\]
and therefore
\[
	\Cl_{l} = \A_{l,S}^\times / \cO_{l,S}^\times, \qquad
	\Cl_{k} = \A_{k,S}^\times / \cO_{k,S}^\times.
\]
The big advantage of working with $\A_{l,S}^\times$ and $\cO_{l,S}$ instead of $\A_l^\times$ and
$l^\times$ is that $\A_{l,S}^\times$ and $\cO_{l,S}$ have finite cohomology groups,
whereas the cohomology groups of $\A_l^\times$ and $l^\times$ are infinite.





\section{The Herbrand quotient of the $S$-ideles}

Let $v$ be a place of $k$ and $\hat v$ a place of $l$ above $v$.
We'll write $D_{\hat v}$ the decomposition group at $\hat v$.

\begin{lemma} \label{lem:semi-local iso coind}
	Then there are isomorphisms
	\[
		\prod_{w | v} l_w^\times
		\cong
		\coind_{D_{\hat v}}^{\Gal(l/k)} l_{\hat v}^\times,
		\qquad
		\prod_{w | v} \cO_w^\times
		\cong
		\coind_{D_{\hat v}}^{\Gal(l/k)} \cO_{\hat v}^\times.
	\]
\end{lemma}

\begin{proof}
	We'll write $x$ for an element of $\prod_{w | v} l_w^\times$ and $x_w$ for it's component in $l_w$.
	Define a map $\Phi : \prod_{w | v} l_w^\times \to (\Gal(l/k) \to l_{\hat v}^\times)$ by
	\[
		\Phi (x)
		=
		(g \mapsto g \bullet x_{g^{-1} \hat v}).
	\]
	Note that for $h \in D_{\hat v}$ we have
	\[
		\Phi (x) (hg) = h \bullet (g \bullet x_{g^{-1} \hat v}) = h \bullet \Phi(x) (g).
	\]
	Therefore $\Phi(x)$ is in the subspace $\coind_{D_v} l_{\hat v}$, and it's
	easy to check that $\Phi$ gives a group isomorphism
	$\prod_{w | v} l_w^\times \cong \coind_{D_v} l_{\hat v}^\times$.
	We'll check that this map intertwines the actions of $G$.
	For $g \in G$, then element $g \bullet x$ has $w$-component
	$g \bullet x_{g^{-1}\bullet w}$.
	This implies
	\[
		\Phi( g \bullet x)(h)
		=	h\bullet (g \bullet x)_{h^{-1} \bullet \hat v}
		=	h g \bullet x_{(hg)^{-1} \bullet \hat v}
		= \Phi (x) (hg).
	\]
	The proof of the isomorphism for $\prod_{w | v} \cO_w^\times$ is similar.
\end{proof}

\begin{lemma} \label{lem:cohomology S-ideles decomp}
	\uses{lem:Shapiro,
		lem:semi-local iso coind,
		lem:unramified units trivial
	}
	There are isomorphisms for all $n > 0$
	\[
		H^n(l/k, \A_{S,l}^\times)
		\cong
		\prod_{v \in S} H^n(l_{\hat v} / k_v, l_{\hat v}^\times).
	\]
\end{lemma}

\begin{proof}
	We note that by \ref{lem:semi-local iso coind} we have
	\[
		\A_{S,l}^\times \cong \prod_{v \in S} \coind_{D_{\hat v}} l_{\hat v}^\times
		\times
		\prod_{v \not\in S} \coind_{D_{\hat v}} \cO_{\hat v}^\times
	\]
	By Shapiro's lemma (\ref{lem:Shapiro}) we have
	\[
		H^n(l/k,\A_{S,l}^\times) \cong
		\prod_{v \in S} H^n(l_{\hat v}/k_v , l_{\hat v}^\times)
		\times
		\prod_{v \not\in S} H^n(l_{\hat v}/k_v , \cO_{\hat v}^\times)
	\]
	For $v \not\in S$, the extension $l_{\hat{v}}/l_v$ is unramified, so
	by \ref{lem:unramified units trivial} $\cO_{\hat v}^\times$ has trivial cohomology.
\end{proof}


\begin{lemma} \label{lem:herbrand S-ideles}
	\uses{lem:cohomology S-ideles decomp,
		lem:herbrand local l*
	}
	If $l/k$ is a cyclic extension then we have
	\[
		h(l/k, \A_{S,l} ^\times )
		=
		\prod_{v \in S} |D_{\hat v}|.
	\]
\end{lemma}

\begin{proof}
	This follows from \ref{lem:cohomology S-ideles decomp} and
	\ref{lem:herbrand local l*}.
\end{proof}





\section{The Herbrand quotient of the $S$-units}

Define the \emph{logarithmic space} $V_S$ to be the following finite dimensional vector space
over the real numbers:
\[
	V_S = \prod_{v \in S} \prod_{w | v} \R.
\]
We consider $V_S$ as a representation of $\Gal(l/k)$, where the Galois action
permutes the places $w$ lying above each $v \in S$.
As a Galois representation we have
\[
	V_S \cong \prod_{v \in S} \ind_{D_{\hat v}}^{\Gal(l/k)} \R.
\]
Contained in $V_S$ we have a lattice $L_S$ consisting of vectors whose components are all in $\Z$.
Here we are using the word ``lattice'' to mean the $\Z$-space of a basis for $V_S$.
We have an isomorphism
\[
	L_S \cong \prod_{v \in S} \ind_{D_{\hat v}}^{\Gal(l/k)} \Z.
\]

\begin{lemma} \label{lem:herbrand L_S}
	\uses{lem:Shapiro,
		lem:H2 cyclic Z
	}
	If $l/k$ is a cyclic extension then
	$h(l/k,L_S) = \prod_{v \in S} |D_{\hat v}|$.
\end{lemma}

\begin{proof}
	This follows from Shapiro's lemma (\ref{lem:Shapiro})
	together with the calculation
	of the cohomology of a cyclic group with values in $\Z$ (\ref{lem:H2 cyclic Z}).
\end{proof}

\begin{lemma} \label{lem:herbrand log lattice}
	\uses{lem:herbrand L_S,
		lem:herbrand ses,
		lem:herbrand finite
	}
	Let $l/k$ be cyclic and let $M$ be any Galois-invariant lattice in $V_S$.
	Then $h(l/k,M) = \prod_{v \in S} |D_{\hat v}|$
\end{lemma}

\begin{proof}
	The representations $M \otimes \Q$ and $L_S \otimes \Q$ have the same character
	(this is just the character of the representation $V_S$).
	Therefore the representations $M \otimes \Q$ and $L_S \otimes \Q$ are isomorphic.
	Hence $M$ is isomorphic to subrepresentation of finite index in $L_S$,
	so by \ref{lem:herbrand ses} and \ref{lem:herbrand finite}
	they have the same Herbrand quotient.
	The result then follows from \ref{lem:herbrand L_S}.
\end{proof}


The vector $(1,1,\ldots,1) \in V_S$ is fixed by all elements of $\Gal(l/k)$, so it spans
a subrepresentation isomorphic to the trivial representation $\Z$.
Recall the we have a logarithmic map
\[
	\log_S : \cO_S^\times \to V_S,
\]
where the $w$-component of $\log_S(x)$ is $\log |x|_w$.
The kernel of $\log_S$ is the finite group of roots of unity in $k$.

\begin{theorem} \label{thm:Dirichlet unit theorem}
	\lean{NumberField.Units.dirichletUnitTheorem.unitLattice_span_eq_top}
	\mathlibok
	$\log_S(\cO_S^\times)$ has zero intersection with $\Span (1,1,\ldots,1)$.
	The direct sum of these subrepresentations is a lattice in $V_S$.
\end{theorem}

\begin{proof}
	\mathlibok
	An equivalent statement is already in Mathlib as
	\texttt{NumberField.Units.dirichletUnitTheorem.unitLattice\_span\_eq\_top}.
\end{proof}

\begin{corollary} \label{lem:herbrand S-units}
	\uses{thm:Dirichlet unit theorem,
		lem:herbrand ses,
		lem:herbrand finite,
		lem:H2 cyclic Z,
		lem:herbrand log lattice
	}
	Let $l/k$ be a cyclic extension. Then
	\[
		h(l/k,\cO_{l,S}^\times) = \frac{\prod_{v\in S} |D_{\hat v}|}{[l:k]}.
	\]
\end{corollary}

\begin{proof}
	Since $\log_S$ has finite kernel, the Herbrand quotient of $\cO_{l,S}^\times$ is
	equal to that of $\log_S(\cO_S^\times)$.
	By Dirichlet's unit theorem, $\log_S(\cO_S^\times) \oplus \Z$ is a lattice in
	$V_S$.
	We know the Herbrand quotient of $\log_S(\cO_S^\times) \oplus \Z$
	from \ref{lem:herbrand log lattice}, and the Herbrand quotient of $\Z$ from \ref{lem:H2 cyclic Z}.
\end{proof}

\begin{corollary} \label{lem:herbrand idele class group}
	\uses{lem:herbrand S-units,
		lem:herbrand S-ideles,
		lem:herbrand ses
	}
	If $l/k$ is cyclic then $h(l/k,\Cl_l) = [l:k]$.
\end{corollary}

\begin{proof}
	Our choice of $S$ implies $\Cl_l \cong \A_{l,S}^\times / \cO_{l,S}^\times$.
	We have calculated the Herbrand quotients of $\A_{l,S}^\times$ and $\cO_{l,S}^\times$
	in \ref{lem:herbrand S-ideles} and \ref{lem:herbrand S-units}.
\end{proof}





\section{Dirichlet Density}

\begin{definition} \label{def:Dirichlet density}
	Let $M$ be a set of primes of $\cO_k$.
	We'll say that $M$ has a \emph{Dirichlet density} $c \in \R$ if
	\[
		\sum_{P \in M} N(P)^{-s} \stackrel{s \to 1+}\sim c  \cdot \log\left(\frac{1}{s-1}\right).
	\]
	where $s$ tends to $1$ through the real numbers $s>1$.
	The symbol $\sim$ denotes asymptotic equivalence, which means that the ratio of the left hand
	side to the right hand side converges to $1$ as $s$ tends to $1$ from above.
	Implied constants may depend of the set $M$ and the field $k$.
\end{definition}


\begin{lemma} \label{lem:Dirichlet density union}
	\uses{def:Dirichlet density}
	Suppose $M_1$ and $M_2$ are disjoint sets of primes of $\cO_l$.
	If two of the sets $M_1, M_2, M_1 \cup M_2$ have a Dirichlet density, then so does the third
	and we have
	\[
		\density(M_1 \cup M_2) = \density(M_1) + \density(M_2).
	\]
\end{lemma}

\begin{proof}
	This is trivial.
\end{proof}

\begin{lemma} \label{lem:Dirichlet density top}
	\uses{def:Dirichlet density}
	The set of all primes of $\cO_k$ has Dirichlet density $1$.
\end{lemma}

\begin{proof}
	Let $P$ be a prime. For $s > 1$ we have
	\[
		\left|N(P)^{-s} - \log\left( \frac{1}{1-N(P)^{-s}}\right)\right| \ll N(P)^{-2s}
	\]
	Since $\sum N(P)^{-2s}$ is bounded in the region $s > 1$ (this follows from the convergence of the
	Dedekind zeta function), we have
	\[
		\sum_P N(P)^{-s}
		\sim
		\sum_P \log\left( \frac{1}{1-N(P)^{-s}} \right)
		=
		\log \zeta_l(s),
	\]
	where $\zeta_l$ is the Dedekind zeta function (\texttt{NumberField.dedekindZeta}).
	By the analytic class number formula (\verb!NumberField.tendsto_sub_one_mul_dedekindZeta_nhdsGT!)
	there is a positive real number $r$ such that
	\[
		\zeta_l(s) = \frac{r}{s-1} + O(1) \qquad (s > 1).
	\]
	This implies
	\[
		\log(\zeta_l(s)) \sim \log\left( \frac{1}{s-1}\right).
	\]
\end{proof}



\begin{lemma} \label{lem:Dirichlet density degree one}
	\uses{lem:Dirichlet density top,
		lem:Dirichlet density union
	}
	The set of primes of $\cO_l$ of degree one has Dirichlet density $1$.
\end{lemma}

\begin{proof}
	Let $M$ be the set of primes of degree larger than one. It's sufficient to
	prove that $M$ has Dirichlet density $0$.
	We have
	\[
		\sum_{P \in M} N(P)^{-s}
		= \sum_{n \in \N} A(n) n^{-s},
	\]
	Where $A$ is the number of primes of degree $>1$ with norm $n$.
	Note that $A(n) \le [l:\Q]$. Also $A(n)=0$ unless $n=m^r$ for positive integer $m$ and
	$1 \le r \le [l:Q]$.
	This implies
	\[
		\sum_{P \in M} N(P)^{-s}
		\le [l:\Q] \sum_{r=2}^{[l:\Q]} \sum_{m =1} ^\infty m^{-rs}
		\le [l:\Q] (\zeta_\Q(2) + \cdots + \zeta_\Q([l:\Q]) ).
	\]
	Since the sum above is bounded on the region $s > 1$,
	the set $M$ has Dirichlet density $0$.
\end{proof}


\begin{lemma} \label{lem:Dirichlet density split}
	\uses{lem:Dirichlet density degree one}
	Let $l/k$ be a finite Galois extension of number fields.
	Then the set of degree $1$ primes of $k$ which split completely in $l$
	has density $\frac{1}{[l:k]}$.
\end{lemma}

\begin{proof}
	Let $M_k$ be the set of degree $1$ primes of $k$ and $M_l$ the set of degree $1$ primes of $l$.
	Every prime $Q$ in $M_l$ lies above some prime $P \in M_k$.
	If there is a prime $Q$ above $P$, then there are precisely $[l:k]$ of them; this happens when $P$
	splits completely in $l$.

	Let $M$ be the set of $P \in M_k$ which split completely in $l$.
	Then we have
	\[
		\sum_{P \in M} N(P)^{-s}
		=
		\frac{1}{[l:k]} \sum_{Q \in M_l} N(Q)^{-s}
		\sim \frac{1}{[l:k]} \log\left( \frac{1}{s-1}\right).
	\]
	Here we have used \ref{lem:Dirichlet density degree one}.
\end{proof}





\section{Some $L$-functions}

\begin{lemma} \label{lem:H0 idele class group finite}
	$H^0_{\Tate}(l/k,\Cl_l)$ is finite.
\end{lemma}

\begin{proof}
	We have
	\begin{align*}
		\Cl_{k} / N(\Cl_l)
		& \cong \A_{k,S}^\times / \cO_{k,S}^\times N(\A_{l,S}^\times) \\
		& \cong \left(\prod_{v \in S} k_v^\times / N(l_{\hat v}^\times)\right) / \cO_{k,S}^\times,
	\end{align*}
	where $\hat v$ is a place of $l$ lying above $v$.
	The result follows because each of the groups $k_v^\times / N(l_w)$ is finite
	(in fact by the local reciprocity isomorphism this is isomorphic to the abelianization of the
	decomposition group at $\hat v$).
\end{proof}

Given any maximal ideal $P$ of $\cO_{k,S}$, we let $\pi_P$ be an idele whose
$P$-component is a uniformizer in $k_P$, and whose other components are all $1$.
The coset of $\pi_P$ in $H^0(l/k,\Cl_l)$ does not depend on the choice of uniformizer since
all the local units at $P$ are local norms from $l_{Q}$ for any $Q|P$
(because $P$ is unramified in $l$).
Equivalently, if we choose a generator $P=(\pi)$ with $\pi \in \cO_{k,S}$
then the coset of $\pi_P$ is the inverse of
the image of $\pi$ in $\prod_{v \in S} k_v^\times / N(l_{\hat v}^\times)$.
This clearly extends to a homomorphism
\[
	\iota : \textrm{non-zero fractional ideals of $\cO_{k,S}$}
	\to
	H^0_{\Tate}(l/k,\Cl_l),
\]
whose kernel is the subgroup of ideals with a generator which is a local norm at $v$ for
all $v \in S$.

\begin{definition} \label{def:L-function}
	\uses{lem:H0 idele class group finite}
	Let $\chi : H^0_{\Tate}(l/k,\Cl_l) \to \C^\times$ be a character.
	For a non-zero ideal $I$ of $\cO_{k,S}$, we shall write $\chi(I)$ in place of $\chi(\iota(I))$.
	We define the $L$-function of $\chi$ by
	\[
		L(s,\chi)
		=
		\sum_{I} \chi(I) \cdot N(I)^{-s}
		=
		\prod_{P} \frac{1}{1-\chi(P) \cdot N(P)^{-s}}.
	\]
	Here $s$ is a complex number with real part greater than $1$; both the product and the series
	converge absolutely in that region.
	In the sum, $I$ ranges over the non-zero ideals of $\cO_{k,S}$, and in the product $P$
	ranges over the maximal ideals of $\cO_{k,S}$.

	If $\chi$ is the trivial character, then $L(s,\chi)$ is
	(up to finitely many Euler factors for primes in $S$)
	equal to the Dedekind zeta function of $k$.
\end{definition}


It's known that $L(s,\chi)$ has a meromorphic continuation to $\C$
and is entire if $\chi$ is non-trivial
(see for example Tate's thesis, which is chapter XV of \cite{cassells frohlich}).
We won't need such a strong result here; we can make do with the following:

\begin{lemma}[Weak lemma] \label{lem:L-function bound}
	\uses{def:L-function}
	If $\chi$ is a non-trivial character then $L(s,\chi)$ is
	bounded on the interval $(1,2)$.
\end{lemma}

\begin{lemma} \label{lem:density bound}
	\uses{lem:L-function bound}
	Let $M$ be a set of primes of $\cO_S$ whose image in $H^0_{\Tate}(l/k,\Cl_l)$ is zero.
	There exists a real number $c$ depending only on the fields $k$ and $l$,
	such that for all $s > 1$ we have:
	\[
		\sum_{p \in M} N(P)^{-s}
		\le \frac{1}{|H^0_{\Tate}(l/k,\Cl_l)|} \log\left(\frac{1}{s-1}\right) + c.
	\]
\end{lemma}


\begin{proof}
	Let $s > 1$. All the series in the following calculation converge absolutely in this region.
	The implied constants in the $O(1)$ terms do not depend on $s$.
	\begin{align*}
		\sum_{P \in M} |N(P)|^{-s}
		&= \frac{1}{|H^0_{\Tate}(l/k,\Cl_l)|} \sum_P \sum_\chi \chi(P) N(P)^{-s}\\
		&= \frac{1}{|H^0_{\Tate}(l/k,\Cl_l)|}
		\sum_P \sum_\chi -\log(1-\chi(P) N(P)^{-s}) + O(1)\\
		&= \frac{1}{|H^0_{\Tate}(l/k,\Cl_l)|}
		\sum_P \sum_\chi -\log |1-\chi(P) N(P)^{-s}| + O(1)\\
		&= \frac{1}{|H^0_{\Tate}(l/k,\Cl_l)|}
		\sum_\chi \sum_P -\log |1-\chi(P) N(P)^{-s}| + O(1)\\
		&= \frac{1}{|H^0_{\Tate}(l/k,\Cl_l)|} \log \left(\prod_\chi |L(s,\chi)| \right) + O(1)\\
		&\le \frac{1}{|H^0_{\Tate}(l/k,\Cl_l)|} \log \left(\frac{1}{s-1}\right) + O(1).
	\end{align*}
	Interchanging the order of summation is justified because the series converge absolutely.
	We need to be slightly careful about which branch of the logarithm we are using here.
	In the expression $\log(1-\chi(P) N(P)^{-s})$ we shall mean the branch which is continuous
	on the ball of radius $1$, centred about $1$.
	The imaginary parts of $\log(1-\chi(P) N(P)^{-s})$ and $\log(1-\bar \chi(P) N(P)^{-s})$
	cancel out; this justifies replacing $\log(1-\chi(P) N(P)^{-s})$ by $\log |1-\chi(P) N(P)^{-s}|$.
\end{proof}

\begin{remark}
	In fact the density of the set $M$ in this lemma is precisely $\frac{1}{|H^0_{\Tate}(l/k,\Cl_l)|}$.
	This can be proved by showing that each $L(s,\chi)$ has a continuation to a neighbourhood
	of $s=1$, and is non-zero at $s=1$.
	However, proving this is more difficult than the weak lemma above, and we only need the inequality
	of the lemma.
\end{remark}






% \begin{lemma}
% 	There is a meromorphic continuation of $L(s,\chi)$
% 	to a neighbourhood of the interval $[\frac{1}{[l:k]},1]$.
% 	the region $\Re s > 1-\frac{1}{[k:\Q]}$.
% 	If $\chi=1$ then there is a simple pole at $s=1$ and no other poles in this region.
% 	If $\chi \ne 1$ then there are no poles.
% \end{lemma}


% We also need the following purely analytic theorem of Landau:

% \begin{theorem}
% 	Suppose we have a Dirichlet series $D(s) = \sum_{n=0}^\infty a_n \cdot n^{-s}$ with
% 	nonnegative real coefficients $a_n$.
% 	Suppose $D(s)$ has absolute convergence abscissa $\sigma$, i.e. it converges absolutely for $\Re s > \sigma$
% 	and not for $\Re s < \sigma$.
% 	Then there is no analytic continuation of $D(s)$ to any neighbourhood of $\sigma$.
% \end{theorem}

% \begin{proof}
% 	By rescaling the coefficients $a_n$ if necessary, we may assume that $\sigma=0$.
% 	Let's suppose that there does exist an nalytic continuation to some neighbourhood of $0$.
% 	This implies that the power series expansion of $D(s)$ about $s=1$ converges on a disk or radius
% 	greater than $1$. Let's assume that this power series converges absolutely one the closed
% 	disk of radius $1+\epsilon$.
% 	This power series expansion is:
% 	\[
% 		D(1+r)
% 		= \sum_{m=0}^\infty \frac{D^{(m)}(1)}{m!} r^m
% 		= \sum_{m=0}^\infty \left(\frac{\sum_{n=1}^\infty a_n (-m \log n )^m n^{-1}}{m!} r^m\right)
% 	\]
% 	Note that if $r < 0$ then every term in the double sum above is positive, so the order of
% 	summation may be swapped.
% 	Substituting $r = -1-\epsilon$ we get
% 	\[
% 		D(-\epsilon)
% 		=\sum_{n=1}^\infty a_n n^{-1}
% 		\left(\sum_{m=0}^\infty  \frac{(-m \log n )^m }{m!} (-1-\epsilon)^m\right).
% 	\]
% 	The inner series in the above expression is the power series expension of $n^{-\epsilon}$,
% 	which converges on the whole complex plane.
% 	We therefore have
% 	\[
% 		D(-\epsilon)
% 		=\sum_{n=1}^\infty a_n n^{-\epsilon}
% 	\]
% 	In particular the series on the right hand side converges,
% 	and the convergence is absolute because all the terms are non-negative real numbers.
% 	This implies absolute convergence of $D(s)$ in the half-place $\Re s > -\epsilon$, contradicting
% 	our assumption.
% \end{proof}


% \begin{theorem}
% 	For each non-trivial character $\chi$ we have $L(s,\chi) \ne 0$.
% \end{theorem}

% \begin{proof}
% 	Consider the following Dirichlet series:
% 	\[
% 		D(s) = \prod_\chi L(s,\chi).
% 	\]
% 	By examining the Euler factors for each prime, we can check that all the coefficients
% 	of $D(s)$ are nonnegative.
% 	Furthermore if $I$ is an ideal of $\cO_S$, then $D(s)$ has a term $N(I)^{-[l:k] s}$.
% 	This implies for real $s$ in the region of convergence
% 	\[
% 		D(s) \ge L(\chi_1,[l:k]s).
% 	\]
% 	If we assume that $L(1,\chi)=0$ for some $\chi$ then $D(s)$ is analytic
% 	in a neighbourhood of $[\frac{1}{[l:k]},1]$.
% 	Hence bu Landau's theorem, the Dirichlet seties for $D(s)$ converges absolutely on that region.
% 	It follows that $L(\chi_1, [l:k]s)$ is converges absolutely on that region.
% 	This contradicts the fact that $L(\chi_1, ds)$ has a pole at $\frac{1}{[l:k]}$
% \end{proof}





\section{The first inequality}

\begin{theorem} \label{thm:first inequality}
	\uses{lem:Dirichlet density split,
		lem:density bound
	}
	For any finite Galois extension $l/k$ be have
	\[
		|H^0_{\Tate}(l/k, \Cl_{l}) | \le [l : k].
	\]
\end{theorem}

\begin{proof}
	Let $M_1$ be the set of degree $1$ primes of $\cO_{k,S}$ which split in $l$ and let $M_2$ be the
	set of primes of $\cO_{k,S}$ whose image in $H^0_{\Tate}(l/k,\Cl_l)$ is zero.
	Given $P \in M_1$, the norm map $l_{\hat v}^\times \to k_v^\times$ is the identity map, and is
	in particular surjective. Hence the image of $P$ in $H^0_{\Tate}(l/k,\Cl_l)$ is zero.
	This shows that $M_1 \subseteq M_2$.
	This implies for $s> 1$:
	\[
		\sum_{P \in M_1} N(P)^{-s}
		\le \sum_{P \in M_2} N(P)^{-s} .
	\]
	By \ref{lem:density bound} we have
	\[
		\sum_{P \in M_1} N(P)^{-s}
		\le
		\frac{1}{|H^0_{\Tate}(l/k,\Cl_l)|} \log \left(\frac{1}{1-s}\right) + O(1).
	\]
	By \ref{lem:Dirichlet density split} the left hand side is asymptotic to
	$\frac{1}{[l:k]}\log \left(\frac{1}{1-s}\right)$.
	Therefore $\frac{1}{[l:k]} \le \frac{1}{|H^0_{\Tate}(l/k,\Cl_l)|}$.
\end{proof}


\begin{corollary} \label{cor:H1 H2 cyclic idele class}
	\uses{thm:first inequality,
		lem:herbrand idele class group
	}
	If $l/k$ is cyclic then $|H^2(l/k, \Cl_{l})| = [l:k]$ and $H^1(l/k, \Cl_{l}) = 0$.
\end{corollary}

\begin{proof}
	This follows immediately from (a) the first inequality, (b) the periodicity of
	the cohomology for a cyclic group, and (c) the calculation of the Herbrand quotient
	of $\Cl_{l}$.
\end{proof}

\begin{theorem} \label{thm:global cohomology bound}
	\uses{cor:H1 H2 cyclic idele class,
		thm:inflation restriction sequence,
		cor:cohomology sub Sylow,
		cor:cohomology G-torsion
	}
	If $l/k$ is any finite Galois extension then $H^1(l/k, \Cl_{l}) \cong 0$
	and $|H^2(l/k, \Cl_{l})| \le [l:k]$.
\end{theorem}


\begin{proof}
	For each prime number $p$ dividing $[l:k]$ we let $k_p$ be the fixed
	field of a Sylow $p$-subgroup $S_p$ of $\Gal(l/k)$.
	By \ref{cor:cohomology sub Sylow}, it's suffient to prove
	\[
		H^1(l/k_p, \Cl_{l}) = 0, \qquad
		|H^2(l/k_p, \Cl_{l})| \le [l:k_p].
	\]
	Since $S_p$ is solvable, this reduces us to the case that $\Gal(l/k)$ is solvable.
	We'll prove the result by induction on $k$ starting with $k=l$ and working downwards
	in cyclic quotients.

	Clearly the result holds for $k=l$.
	Assume the result for a subfield $m$ of $l$ and let $m/k$ be cyclic.
	We have an inflation restriction sequence:
	\[
		0 \to H^1(m/k, \Cl_{m}) \to H^1(l/k, \Cl_{l}) \to H^1(l/m,\Cl_{l}).
	\]
	The first term is zero by \ref{cor:H1 H2 cyclic idele class} and the last term is zero by
	the inductive hypothesis.
	Therefore $H^1(l/k,\Cl_l) \cong 0$.

	Since $H^1(l/m,\Cl_l)$ is assumed to be zero, we also have an inflation-restriction sequence
	in dimension 2:
	\[
		0 \to H^2(m/k, \Cl_m) \to H^2(l/k, \Cl_l) \to H^2(l/m,\Cl_l).
	\]
	By the inductive hypothesis we have $|H^2(l/m,\Cl_l)| \le [l:m]$
	and by \ref{cor:H1 H2 cyclic idele class} we have $|H^2(m/k, \Cl_m)| = [m : k]$.
	It follows that $|H^2(l/k,\Cl_l)| \le [l : k]$.
\end{proof}


To complete the construction of fundamental classes and the reciprocity isomorphism,
we need only show that there is an element in $H^2(l/k,\Cl_l)$ of order $[l:k]$.
Such an element is constructed first for a cyclic cyclotomic extension $l'/k$ with the same degree
as $l/k$.
It's then shown that the inflation of such a class to $ll'/k$ must split on $ll'/l$,
and must therefore be the inflation of an element of order $[l:k]$ in $H^2(l/k,\Cl_l)$.
